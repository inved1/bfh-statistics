%Simon Krenger: Wahrscheinlichkeitsrechnung und Statistik (BFH, FS2014)
\documentclass{report}
\usepackage{amsmath}
\usepackage{amsthm}
\usepackage{amssymb}
\usepackage[utf8]{inputenc} 
\usepackage{graphicx}
\usepackage{mathtools}
\usepackage{floatrow}

\usepackage{color}
\definecolor{red}{rgb}{1,0,0}

\newtheorem{mydef}{Definition}
\newtheorem{myexample}{Beispiel}
\newtheorem{myproof}{Beweis}
\newtheorem{axiom}{Axiom}
\newtheorem{satz}{Satz}

\title{Wahrscheinlichkeitsrechung und Statistik}
\author{Simon Krenger}

\begin{document}
\maketitle
\chapter{Wahrscheinlichkeitsrechnung}
\section{Definitionen}
Wir führen ein Experiment wie
\begin{itemize}
\item werfen von 2 Münzen
\item werfen von 3 Würfeln
\item ziehen einer Zahl aus einer Urne
\end{itemize}
durch und fragen nach möglichen Ereignissen. Also schreiben wir diese als Menge auf
\begin{equation}
M = \{ KK, KZ, ZK, ZZ \}
\end{equation}
oder
\begin{equation}
M = \{ KK, KZ, ZZ \}
\end{equation}
\begin{mydef}
Die Menge
\begin{equation}
\Omega := \{ \omega_1, \omega_2, \omega_3, ..., \omega_n \}
\end{equation}
heisst \underline{Stichprobenraum} (Ereignisraum), wenn jedem Versuchsausgang höchstens ein Element $\omega_i$ aus $\Omega$ zugeordnet ist.
\end{mydef}
Beim Werfen eines Würfels sind
\begin{itemize}
\item $\Omega_1 = \{ gerade, \quad ungerade \}$
\item $\Omega_2 = \{ 1,2,3,4,5,6 \}$
\item $\Omega_3 = \{ 4, \quad keine \quad 4 \}$
\end{itemize}
mögliche Stichprobenräume.\\\\
Wie gross ist die Wahrscheinlichkeit beim Werfen von 2 (idealen) Würfeln zwei Sechsen zu erhalten?
Als Stichprobenräume können wir
\begin{align*}
 \Omega_1 =& \{ (1/1), (1/2), (1/3), (1/4), (1/5), (1/6), \\
 \quad & (2/2), (2/3), ..., (2/6), \\
 \quad & (3/3), (3/4), ..., (3/6), \\
 \quad & (4/4), (4/5), (4/6), \\
 \quad & (5/5), (5/6), \\
 \quad & (6/6) \}
\end{align*}
wählen. Wir unterscheiden also z.B. $(2/3)$ und $(3/2)$ nicht.\\\\
Auch
\begin{align*}
 \Omega_1 =& \{ (1/1), (1/2), (1/3), (1/4), (1/5), (1/6), \\
 \quad & (2/1), (2/2), (2/3), ..., (2/6), \\
 \quad & (3/1), (3/2), (3/3), ..., (3/6), \\
 \quad & ... \\
 \quad & (6/1), (6/2), (6/3), ..., (6/6) \}
\end{align*}
ist ein möglicher Stichprobenraum.\\\\
Im ersten Fall ist $|\Omega_1| = 21$ und im zweiten Fall ist $|\Omega_2| = 36$. Sind alle \underline{Ereignisse gleichwahrscheinlich}, so ist die Wahrscheinlichkeit zwei 6 zu würfeln
\begin{itemize}
\item im 1. Fall $ p = \frac{1}{21} $
\item im 2. Fall $ p = \frac{1}{36} $
\end{itemize}
Welches Modell entspricht der Praxis? (Im Praxisversuch finden wir, dass $\frac{1}{36}$, also der zweite Fall, der Praxis entspricht)\\\\
\begin{mydef}
Jede Teilmenge von $\Omega$ heisst \underline{Ereignis}. Die leere Menge $\emptyset$ heisst \underline{unmögliches Ereignis} und $\Omega$ heisst \underline{sicheres Ereignis}.\\
Enthält ein Ereignis $E = \{ a \}$ nur ein einziges Element, so heisst $E$ ein \underline{Elementarereignis}.
\end{mydef}
\begin{myexample}
Beim Werfen von 2 Würfeln ist
\begin{equation}
\Omega = \{ (1/1), (1/2), ..., (6/6) \}
\end{equation}
und
\end{myexample}
\end{document}