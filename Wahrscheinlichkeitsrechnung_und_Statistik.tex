%Simon Krenger: Wahrscheinlichkeitsrechnung und Statistik (BFH, FS2014)
\documentclass{report}
\usepackage{amsmath}
\usepackage{amsthm}
\usepackage{amssymb}
\usepackage[utf8]{inputenc} 
\usepackage{graphicx}
\usepackage{mathtools}
\usepackage{floatrow}
\usepackage{enumerate}

\usepackage{color}
\definecolor{red}{rgb}{1,0,0}
\definecolor{green}{rgb}{0,1,0}
\definecolor{blue}{rgb}{0,0,1}

\newtheorem{mydef}{Definition}
\newtheorem{myexample}{Beispiel}
\newtheorem{myproof}{Beweis}
\newtheorem{axiom}{Axiom}
\newtheorem{satz}{Satz}

\title{Wahrscheinlichkeitsrechung und Statistik}
\author{Simon Krenger}

\begin{document}
\maketitle
\chapter{Wahrscheinlichkeitsrechnung}
\section{Definitionen}
Wir führen ein Experiment wie
\begin{itemize}
\item werfen von 2 Münzen
\item werfen von 3 Würfeln
\item ziehen einer Zahl aus einer Urne
\end{itemize}
durch und fragen nach möglichen Ereignissen. Also schreiben wir diese als Menge auf
\begin{equation}
M = \{ KK, KZ, ZK, ZZ \}
\end{equation}
oder
\begin{equation}
M = \{ KK, KZ, ZZ \}
\end{equation}
\begin{mydef}
Die Menge
\begin{equation}
\Omega := \{ \omega_1, \omega_2, \omega_3, ..., \omega_n \}
\end{equation}
heisst \underline{Stichprobenraum} (Ereignisraum), wenn jedem Versuchsausgang höchstens ein Element $\omega_i$ aus $\Omega$ zugeordnet ist.
\end{mydef}
Beim Werfen eines Würfels sind
\begin{itemize}
\item $\Omega_1 = \{ gerade, \quad ungerade \}$
\item $\Omega_2 = \{ 1,2,3,4,5,6 \}$
\item $\Omega_3 = \{ 4, \quad keine \quad 4 \}$
\end{itemize}
mögliche Stichprobenräume.\\\\
Wie gross ist die Wahrscheinlichkeit beim Werfen von 2 (idealen) Würfeln zwei Sechsen zu erhalten?
Als Stichprobenräume können wir
\begin{align*}
 \Omega_1 =& \{ (1/1), (1/2), (1/3), (1/4), (1/5), (1/6), \\
 \quad & (2/2), (2/3), ..., (2/6), \\
 \quad & (3/3), (3/4), ..., (3/6), \\
 \quad & (4/4), (4/5), (4/6), \\
 \quad & (5/5), (5/6), \\
 \quad & (6/6) \}
\end{align*}
wählen. Wir unterscheiden also z.B. $(2/3)$ und $(3/2)$ nicht.\\\\
Auch
\begin{align*}
 \Omega_1 =& \{ (1/1), (1/2), (1/3), (1/4), (1/5), (1/6), \\
 \quad & (2/1), (2/2), (2/3), ..., (2/6), \\
 \quad & (3/1), (3/2), (3/3), ..., (3/6), \\
 \quad & ... \\
 \quad & (6/1), (6/2), (6/3), ..., (6/6) \}
\end{align*}
ist ein möglicher Stichprobenraum.\\\\
Im ersten Fall ist $|\Omega_1| = 21$ und im zweiten Fall ist $|\Omega_2| = 36$. Sind alle \underline{Ereignisse gleichwahrscheinlich}, so ist die Wahrscheinlichkeit zwei 6 zu würfeln
\begin{itemize}
\item im 1. Fall $ p = \frac{1}{21} $
\item im 2. Fall $ p = \frac{1}{36} $
\end{itemize}
Welches Modell entspricht der Praxis? (Im Praxisversuch finden wir, dass $\frac{1}{36}$, also der zweite Fall, der Praxis entspricht)\\\\
\begin{mydef}
Jede Teilmenge von $\Omega$ heisst \underline{Ereignis}. Die leere Menge $\emptyset$ heisst \underline{unmögliches Ereignis} und $\Omega$ heisst \underline{sicheres Ereignis}.\\
Enthält ein Ereignis $E = \{ a \}$ nur ein einziges Element, so heisst $E$ ein \underline{Elementarereignis}.
\end{mydef}
\begin{myexample}
Beim Werfen von 2 Würfeln ist
\begin{equation}
\Omega = \{ (1/1), (1/2), ..., (6/6) \}
\end{equation}
und\\\\
\textbf{A: zwei Sechsen würfeln,}\\
also $A = \{ (6/6) \}$ ein Elementarereignis.\\
\textbf{B: nur Primzahlen würfeln,}\\
also $B = \{ (2/2), (2/3), (2/5), (3/2), (3/3), (3/5), (5/2), (5/3), (5/5) \}$\\
\textbf{C: Augensumme 9 würfeln,}\\
also $C = \{ (3/6), (4/5), (5/4), (6/3) \}$\\
\textbf{D: zweimal 7 würfeln,}\\
also $D = \emptyset $\\\\
sind mögliche Ereignisse.
\end{myexample}
Welches ist nun die Wahrscheinlichkeit für eines dieser Ereignisse?

\section{Definition von Laplace}
\textit{(Pierre Simon Laplace, 1749 bis 1827, Paris)}\\\\
Hat in $\Omega = \{ A_1, A_2, A_3, ..., A_n \}$ jedes Ereignis die gleiche Wahrscheinlichkeit, so ist die Wahrscheinlichkeit für ein Ereignis $E = A_1 \cup A_2 \cup ... \cup A_k (k \leq n)$ bestimmt durch
\begin{equation}
P(E) = \frac{|E|}{|\Omega|}
\end{equation}
Es werden also die günstigen Fälle durch die möglichen Fälle dividiert:
\begin{quote}
\textit{( \lq\lq{}günstige Fälle\rq\rq{} \quad : \quad \lq\lq{}mögliche Fälle\rq\rq{} )}
\end{quote}
Für die oben genannten Ereignisse ist also $|\Omega| = 36$,
\begin{equation}
|A| = 1 \quad \longrightarrow \quad p = \frac{1}{36}
\end{equation}
\begin{equation}
|B| = 9 \quad \longrightarrow \quad p = \frac{9}{36} = \frac{1}{4}
\end{equation}
\begin{equation}
|C| = 4 \quad \longrightarrow \quad p = \frac{4}{36} = \frac{1}{9}
\end{equation}
\begin{equation}
|D| = 0 \quad \longrightarrow \quad p = 0
\end{equation}
\begin{mydef}
\underline{1. Zählprinzip}: Es gibt $n^k$ Möglichkeiten um $n$ Elemente auf $k$ Plätze (in Gruppen zu $k$ Elementen) zu verteilen.
\end{mydef}
\begin{itemize}
\item 2 Würfel: $|\Omega| = 6^2$
\item 3 Würfel: $|\Omega| = 6^3$
\item 3 Münzen: $|\Omega| = 2^3$
\item 4 Münzen: $|\Omega| = 2^4$
\end{itemize}
\begin{myexample}
Wie gross ist die Wahrscheinlichkeit
\begin{enumerate}
\item Beim Werfen von 4 Münzen dreimal Zahl und einmal Kopf zu erhalten?\\
\begin{align*}
 |\Omega| &= 2^4 = 16\\
 A &= \{ ZZZK, ZZKZ, ZKZZ, KZZZ \}\\
 \rightarrow |A| &= 4
\end{align*}
\item Beim Werfen von 5 Würfeln vier mal eine 1 und einmal eine 6 zu erhalten?
\begin{align*}
 |\Omega| &= 6^5\\
 B &= \{ 11116, 11161, 11611, 16111, 61111 \}\\
 \rightarrow |B| &= 5
\end{align*}
\item Beim Toto zu gewinnen?\\
Bei 13 Spielen ist
\begin{quote}
$|\Omega| = 3^{13} = 1594323$
\end{quote}
und einer dieser Tipps ist richtig, also
\begin{quote}
$p = \frac{1}{3^{13}} = 6.27 * 10^{-7}$
\end{quote}
\end{enumerate}
\end{myexample}

\section{Axiome von Kolmogorow}
(Andrej Nikolajewitsch Kolmogorow, 25. April 1903 bis 20. Oktober 1987, Moskau)\\\\
Betrachten wir den Stichprobenraum $\Omega$, so sind bekanntlich alle Teilmengen von $\Omega$ die Ereignisse. Die Menge aller Teilmengen ist die Potenzmenge $P(\Omega)$.
\begin{mydef}
Wir nennen $P(\Omega)$ den \underline{Ereignisraum}.
\end{mydef}
\begin{axiom} Die Wahrscheinlichkeit $p$ ist eine Funktion über $P(\Omega)$ mit den Eigenschaften
\begin{enumerate}
\item $0 \leq P(A) \leq 1$ für alle $A \in P(\Omega)$
\item $P(\Omega) = 1$
\item $A, B \in P(\Omega)$ und $ A \cap B = \emptyset$, so $P(A \cup B) = P(A) + P(B)$
\end{enumerate}
\end{axiom}
Axiom 3 sagt uns, dass die Ereignisse A und B \underline{unvereinbar} sind.\\
Für $P(A \cup B)$ sagen wir \lq\lq{}Die Wahrscheinlichkeit, dass A \underline{oder} B eintrifft...\rq\rq{}\\\\
\begin{myexample}
\begin{enumerate}
\item
Die Wahrscheinlichkeit, beim Werfen eines Würfels eine 2 oder eine ungerade Zahl zu erhalten
\begin{equation}
A = \{2\} \longrightarrow |A| = 1 \longrightarrow P(A) = \frac{1}{6}
\end{equation}
\begin{equation}
B = \{1,3,5\} \longrightarrow |B| = 3 \longrightarrow P(B) = \frac{3}{6}
\end{equation}
Es ist $A \cap B = \emptyset$, also
\begin{equation}
P(A \cup B) = \frac{1}{6} + \frac{3}{6} = \frac{4}{6} = \frac{2}{3}
\end{equation}
\item
Wie gross ist die Wahrscheinlichkeit beim Würfeln von 2 Würfeln die Augensumme 10 zu erhalten?
\begin{quote}
10 = 4 + 6 = 6 + 4 oder 10 = 5 + 5
\end{quote}
\begin{equation}
A = \{(4/6), (6/4)\} \longrightarrow P(A) = \frac{2}{36}
\end{equation}
\begin{equation}
B = \{(5/5)\} \longrightarrow P(B) = \frac{1}{36}
\end{equation}
Es ist $A \cap B = \emptyset$, also
\begin{equation}
P(A \cup B) = \frac{2}{36} + \frac{1}{36} = \frac{3}{36} = \frac{1}{12}
\end{equation}
\end{enumerate}
\end{myexample}
Wie ist es nun, wenn $A \cap B \neq \emptyset$?
Es ist $A \cup B = A \cup (\overline{A} \cap B)$ und $B = (A \cap B) \cup (\overline{A} \cap B)$.
\\\\TODO: Grafik\\\\
Wir sehen, dass
\begin{quote}
$A$ und $\overline{A} \cap B$ disjunkt und $A \cap B$ und $\overline{A} \cap B$ disjunkt sind. Nach Axiom 3 ist also
\begin{equation}
P(A \cup B) = P(A) + P(\overline{A} \cap B)
\end{equation}
\begin{equation}
P(B) = P(A \cap B) + P(\overline{A} \cap B)
\end{equation}
\begin{equation}
P(A \cup B) - P(B) = P(A) - P(A \cap B)
\end{equation}
und damit
\begin{equation}
P(A \cup B) = P(A) + P(B) - P(A \cap B)
\end{equation}
\end{quote}
\begin{myexample}
Wahrscheinlichkeit, dass beim Werfen von 2 Würfeln beide Zahlen ungerade oder gleich sind.
\begin{align*}
 A =& \{(1/1),(1/3),(1/5), ..., (5/5)\} \\
 B =& \{(1/1),(2/2),(3/3),(4/4),(5/5),(6/6)\}\\
 \rightarrow A \cap B =& \{(1/1),(3/3),(5/5)\}
\end{align*}
Es ist $A = \{1,3,5\} \times \{1,3,5\}$ also
\begin{equation}
A = 3 \cdot 3 = 9 \longrightarrow P(A) = \frac{9}{36}
\end{equation}
Also
\begin{equation}
P(A \cup B) = \frac{9}{36} + \frac{6}{36} - \frac{3}{36} = \frac{12}{36} = \frac{1}{3}
\end{equation}
\end{myexample}
\begin{mydef}
Zwei Ereignisse heissen (stochastisch) \underline{unabhängig}, wenn
\begin{equation}
P(A \cap B) = P(A) \cdot P(B)
\end{equation}
\end{mydef}
Wir sagen für $P(A \cap B)$, die Wahrscheinlichkeit, dass $A$ \underline{und} $B$ eintreten.
\begin{myexample}Beispiele zur stochastischen Unabhängigkeit:
\begin{enumerate}
\item
Zwei Münzen werden geworfen und
\begin{quote}
A: Höchstens einmal \lq\lq{}Zahl\rq\rq{}\\
B: Jede Seite mindestens einmal
\end{quote}
Sind die Ereignisse unabhängig?
\begin{align*}
 A =& \{KK, KZ, ZK\} \longrightarrow P(A) = \frac{3}{4}\\
 B =& \{KZ, ZK\} \longrightarrow P(B) = \frac{2}{4}\\
A \cap B =& \{(KZ, ZK)\} \longrightarrow P(A \cap B) = \frac{2}{4}
\end{align*}
also $P(A) \cdot P(B) = \frac{3}{4} \cdot \frac{1}{2} = \frac{3}{8}$ und $P(A \cap B) = \frac{1}{2}$, also
\begin{equation}
P(A) \cdot P(B) \neq P(A \cap B)
\end{equation}
also sind die Ereignisse A  und B nicht unabhängig
\item
2 Würfel werfen und
\begin{quote}
A: Augensumme ist ungerade\\
B: Augensumme ist gerade
\end{quote}
\begin{align*}
 A =& \{3,5,7,9,11\} \longrightarrow P(A) = \frac{18}{36}\\
 B =& \{2,4,6,8,10,12\} \longrightarrow P(B) = \frac{18}{36}
\end{align*}
somit $A \cap B = \emptyset$, dann ist
\begin{equation}
P(A) \cdot P(B) = \frac{1}{2} \cdot \frac{1}{2} = \frac{1}{4}
\end{equation}
und $P(A \cap B) = 0$, also sind die Ereignisse nicht unabhängig!
\end{enumerate}
\end{myexample}

\section{Baumdiagramm, Urnenmodell}
Wenn unser Experiment aus mehreren Schritten besteht, so eignet sich das Baumdiagramm, um die Lösung zu finden.
\begin{myexample}
In einer Urne sind 4 Zettel mit den Buchstaben $A, N, B, E$. Wie gross ist die Wahrscheinlichkeit, beim Ziehen dieser Zettel \underline{mit Zurücklegen}, dass Wort $ANNA$ zu erhalten?
\\\\TODO: Grafik\\\\
Wir benötigen also
\begin{quote}
A und A und N und N\\
oder\\
A und N und A und N\\
oder ...
\end{quote}
Genauso berechnen wir die Wahrscheinlichkeit: Längs eines Weges multiplizieren wir und alle diese so erhaltenen Wahrscheinlichkeiten addieren wir dann:
\begin{equation}
p = \frac{1}{4} \cdot \frac{1}{4} \cdot \frac{1}{4} \cdot \frac{1}{4} + \frac{1}{4} \cdot \frac{1}{4} \cdot \frac{1}{4} \cdot \frac{1}{4} + ... = 6 \cdot \frac{1}{4^4} = \frac{3}{128}
\end{equation}
\end{myexample}
\subsection{Urnenmodell}
Allgemein betrachten wir eine Urne, in welcher verschiedenartige Kugeln sind. Ziehen wir nun Kugeln, so können wir dies entweder \underline{mit Zurücklegen} oder \underline{ohne Zurücklegen} machen.
\begin{myexample}
In einer Urne sind 4 blaue und 3 rote Kugeln. Wie gross ist die Wahrscheinlichkeit, dass wir beim Ziehen
\begin{enumerate}
\item
mit Zurücklegen
\item
ohne Zurücklegen
\end{enumerate}
2 blaue und eine rote Kugel erhalten?
\begin{enumerate}
\item
TODO: Grafik mit Zurücklegen\\\\
\begin{equation}
p = \frac{4}{7} \cdot \frac{4}{7} \cdot \frac{3}{7} + \frac{4}{7} \cdot \frac{4}{7} \cdot \frac{3}{7} + \frac{4}{7} \cdot \frac{4}{7} \cdot \frac{3}{7}
\end{equation}
\begin{equation}
p = 3 (\frac{4}{7})^2(\frac{3}{7})^1 = 3 \cdot \frac{16}{49} \cdot \frac{3}{7} = \frac{144}{343}
\end{equation}
\item
TODO: Grafik\\\\
\begin{equation}
p = \frac{4}{7} \cdot \frac{3}{6} \cdot \frac{3}{5} + \frac{4}{7} \cdot \frac{3}{6} \cdot \frac{3}{5} +\frac{4}{7} \cdot \frac{3}{6} \cdot \frac{3}{5}
\end{equation}
\begin{equation}
p = 3 \cdot \frac{4}{7} \cdot \frac{32}{5 \cdot 6} = \frac{108}{210} = \frac{54}{105} = \frac{18}{35}
\end{equation}
\end{enumerate}
\end{myexample}
Das Baumdiagramm hilft uns auch, wenn wir fragen, wieviele Möglichkeiten es gibt aus einer Menge mit $n$ Elementen eine Teilmenge mit $k$ Elementen $(k \leq n)$ auszuwählen und die \underline{Reihenfolge wesentlich} ist:
\begin{equation}
abc \neq bac \neq cab
\end{equation}
\\\\TODO: Grafik\\\\
Für die Wahl des 1. Elementes haben wir 5 Möglichkeiten, für die Wahl des 2. Elementes haben wir 4 Möglichkeiten und dann noch 3 Möglichkeiten für das 3. Element also
\begin{equation}
5 \cdot 4 \cdot 3 = 60 \mbox{ Möglichkeiten}
\end{equation}
\begin{myexample}
Wie gross ist die Wahrscheinlichkeit, beim Werfen von 3 Würfeln lauter verschiedener Zahlen zu erhalten?
\begin{quote}
Günstige Fälle: $6 \cdot 5 \cdot 4 \cdot 3$\\
Mögliche Fälle: $6^4$
\end{quote}
Also
\begin{equation}
p = \frac{6 \cdot 5 \cdot 4 \cdot 3}{6^4} = \frac{5 \cdot 4 \cdot 3}{6^3} = \frac{5 \cdot 4}{6^2 \cdot 2} = \frac{5}{18}
\end{equation}
\end{myexample}
\begin{mydef}
\underline{2. Zählprinzip:} Wählen wir $k$ aus $n$ Elementen, so gibt es also
\end{mydef}
\begin{equation}
n (n-1) (n-2) ... (n-k+1)
\end{equation}
Möglichkeiten, wenn die Reihenfolge wesentlich ist.
\begin{myexample}
Wieviele fünfstellige Zahlen
\begin{enumerate}[(a)]
\item
mit lauter ungeraden Ziffern
\item
mit lauter verschiedener geraden Ziffern
\end{enumerate}
gibt es?
\begin{enumerate}[(a)]
\item
1. Zählprinzip: Für jede Ziffer eine der 5 Zahlen $1,3,5,7,9$, also $5^5$ Möglichkeiten.
\item
2. Zählprinzip: An erster Stelle eine der Zahlen $2,4,6,8$, an zweiter Stelle $0,2,4,6,8$, aber nicht dieselbe wie schon an erster Stelle etc. Also
\begin{equation}
4 \cdot 4 \cdot 3 \cdot 2 \cdot 1 = 96 \mbox{ Möglichkeiten}
\end{equation}
\end{enumerate}
\end{myexample}

\section{Gegenwahrscheinlichkeit}
Oft ist es einfacher, die Wahrscheinlichkeit zu berechnen, dass A nicht eintrifft.
\begin{myexample}
Wie gross ist die Wahrscheinlichkeit beim Würfeln von 3 Münzen mindestens einmal \lq\lq{}Zahl\rq\rq{} zu erhalten?
\end{myexample}
Mindestens einmal \lq\lq{}Zahl\rq\rq{} bedeutet:
\begin{quote}
Einmal, zweimal oder dreimal \lq\lq{}Zahl\rq\rq{}
\end{quote}
,also ist
\begin{equation}
A = \{ZKK, ..., ZZK, ..., ZZZ \}
\end{equation}
und das \underline{Gegenereignis} ist
\begin{equation}
\overline{A} = \{KKK\}
\end{equation}
\\\\TODO: Grafik\\\\
Es ist also $A \cup \overline{A} = \Omega$ und $A \cap \overline{A} = \emptyset$. Nach Axiom (3) ist also
\begin{align*}
 P(A \cup \overline{A}) =& P(A) + P(\overline{A})\\
 \longrightarrow P(\Omega) =& P(A) + P(\overline{A})\\
 1 =& P(A) + P(\overline{A})
\end{align*}
\begin{mydef}
Es ist also
\begin{equation}
P(\overline{A}) = 1 - P(A)
\end{equation}
heisst die Gegenwahrscheinlichkeit von A.
\end{mydef}
Im Beispiel ist also $P(\overline{A}) = \frac{1}{8}$ und damit ist $P(A) = 1 - \frac{1}{8} = \frac{7}{8}$.
\begin{myexample}
Wie gross ist die Wahrscheinlichkeit beim Werfen von 3 Würfeln nicht 3 gleiche Zahlen zu erhalten?
\begin{align*}
A =& \{(1/1/1),(2/2/2), ..., (6/6/6)\}\\
\longrightarrow |A| =& 6 \mbox{ und } |\Omega| = 6^3
\end{align*}
ist
\begin{equation}
P(\overline{A}) = 1 - P(A) = 1 - \frac{6}{6^3} = 1 - \frac{1}{6^2} = \frac{35}{36}
\end{equation}
\end{myexample}
Wir wählen in Zukunft die Schreibweise
\begin{enumerate}
\item
$P(S=15)$ für die Wahrscheinlichkeit, dass die Augensumme 15 ist.
\item
$P(Z\geq1)$ für die Wahrscheinlichkeit, dass \underline{mindestens} einmal \lq\lq{}Zahl\rq\rq{} auftritt.
\item
$P(K\leq3)$ für die Warhscheinlichkeit, dass \underline{höchstens} dreimal \lq\lq{}Kopf\rq\rq{} vorkommt.
\end{enumerate}
\subsection{Geburtstagsproblem}
Wie gross ist die Wahrscheinlichkeit, dass unter $n$ Personen $(n \leq 366)$ mindestens 2 am gleichen Tag Geburtstag haben?\\\\
Mögliche Fälle: An erster Stelle kann eines der 365 Daten stehen, an zweiter, dritter, vierter, ... Stelle ebenso. Also gibt es
\begin{equation}
365^n \mbox{ mögliche Fälle}
\end{equation}
Günstige Fälle: Es ist einfacher, diejenigen Mengen zu suchen, die lauter verschiedene Daten enthalten.\\\\
An erster Stelle sind dann 365, an zweiter Stelle noch 364, an dritter Stelle noch 363, ... Daten möglich. Also gibt es für das Gegenereignis
\begin{equation}
365 \cdot 364 \cdot 363 \cdot ... \cdot (365-n+1)
\end{equation}
günstige Fälle. Damit
\begin{equation}
p = 1 - \frac{365 \cdot 364 \cdot 363 \cdot ... \cdot (365-n + 1)}{365^n}
\end{equation}
\begin{myexample}
Ein Schütze trifft die Scheibe mit einer Wahrscheinlichkeit $p = \frac{5}{6}$. Wie gross ist die Wahrscheinlichkeit, dass er bei 4 Schüssen mindestens einmal trifft?\\\\
Mindestens einmal bedeutet $P(T=1) + P(T=2) + P(...$\\
Gegenereignis: Keinmal treffen, d.h. $P(T=0) = \frac{1}{6}$\\\\
So ist dann
\begin{equation}
P(T\geq1) = 1 - P(T=0)
\end{equation}
in 4 Schüssen, also
\begin{align*}
P(T \geq 1) =& 1 - (\frac{1}{6})^4\\
=& 1 - \frac{1}{1296} = \frac{1293}{1296}
\end{align*}
Ein anderer Schütze trifft die Scheibe mit $p = \frac{3}{5}$. Wie oft muss er schiessen, um mit einer Wahrscheinlichkeit von mindestens $0.99$ einmal zu treffen?
\begin{equation}
P(T \geq 1) = 1 - P(T=0)
\end{equation}
In x Schüssen also
\begin{align*}
P(T \geq 1) =& 1 - (\frac{2}{5})^x\\
\longrightarrow 1 - (\frac{2}{5})^x \geq & 0.99\\
\longrightarrow 1 - (\frac{2}{5})^x \geq & \frac{99}{100}\\
\longrightarrow \frac{1}{100} \geq & (\frac{2}{5})^x \\
\longrightarrow \log{\frac{1}{100}} \geq & x \cdot \log{\frac{2}{5}} \\
\longrightarrow -2 \geq & x \cdot \log{\frac{2}{5}} \\
\longrightarrow -\frac{2}{log{\frac{2}{5}}} \geq & x
\end{align*}
, was falsch ist, denn $\log{\frac{2}{5}}$ ist negativ! Also
\begin{equation}
\frac{-2}{\log{\frac{2}{5}}} \geq x \approx 5.03
\end{equation}
Also mindestens 6 Schüsse.
\end{myexample}

\section{Kombinatorik}
Wie gross ist die Wahrscheinlichkeit beim Werfen von 6 Würfeln 6 verschiedene Zahlen zu erhalten?
\begin{quote}
Mögliche Fälle: $|\Omega| = 6^6$\\
Günstige Fälle: $6 \cdot 5 \cdot 4 \cdot 3 \cdot 2 \cdot 1$
\end{quote}
also
\begin{equation}
p =\frac{6 \cdot 5 \cdot 4 \cdot 3 \cdot 2 \cdot 1}{6^6} = \frac{120}{6^5} = \frac{20}{6^4} = \frac{10}{3 \cdot 6^3} = \frac{5}{3 \cdot 3 \cdot 6^2} = \frac{5}{324}
\end{equation}
Wieviele sinnvolle oder sinnlose Worte können wir mit den Buchstaben von $DIENSTAG$ bilden?\\
\begin{quote}
An erster Stelle stehen 8 Möglichkeiten,\\
an zweiter Stelle stehen 7 Möglichkeiten, ...
\end{quote}
also
\begin{equation}
8 \cdot 7 \cdot 6 \cdot 5 \cdot 4 \cdot 3 \cdot 2 \cdot 1
\end{equation}
Worte.
\begin{mydef}
Ist $n \in \mathbb{N}$, so heisst
\begin{equation}
n! := n \cdot (n-1) \cdot (n-2) \cdot ... \cdot 3 \cdot 2 \cdot 1
\end{equation}
die \underline{Fakultät} von $n$.\\
Ausserdem ist $0! = 1$.
\end{mydef}
Die Fakultäten wachsen sehr schnell:
\begin{align*}
4! = & 24\\
5! = & 120\\
6! = & 720\\
7! = & 5040
\end{align*}
Wieviele Nullen stehen am Schluss von $100!$ ? Eine 0 entsteht durch Multiplikation mit $10$ und $10 = 2 \cdot 5$. Es ist
\begin{equation}
100! = 100 \cdot 99 \cdot 98 \cdot 97 \cdot ... \cdot 3 \cdot 2 \cdot 1
\end{equation}
und durch 5 dividierbar sind
\begin{quote}
5, 10, 15, 20, 25, 30, ..., 95, 100
\end{quote}
Das sind 20 Zahlen, aber
\begin{equation}
25 = 5 \cdot 5,\quad 50 = 2 \cdot 5 \cdot 5 \mbox{ und } 75 = 3 \cdot 5 \cdot 5
\end{equation}
somit sind 24 Nullen am Schluss von $100!$.\\
Aufpassen:
\begin{equation}
3! \cdot 4! \neq 12! \mbox{ und } \frac{16!}{4!} \neq 2!
\end{equation}
\begin{mydef}
\underline{3. Zählprinzip:} Es gibt $n!$ Möglichkeiten um $n$ Elemente auf $n$ Plätze zu verteilen.
\end{mydef}
Wieviele Sitzordnungen gibt es in der I3q?\\
Es gibt $19!$ Sitzordnungen (für 19 Sitzordnungen auf 19 Stühlen). Wechseln wir jede Minute, so dauert es $19!$ Minuten, also $\approx$ 1 Trillion Jahre.\\\\
Suchen wir alle Worte, die sich mit
\begin{quote}
$STRASSE$
\end{quote}
bilden lassen, so wären dies
\begin{quote}
$7!$ Worte
\end{quote}
aber
\begin{align*}
 & \textcolor{blue}{S}TRA\textcolor{red}{S}\textcolor{green}{S}E\\
= & \textcolor{green}{S}TRA\textcolor{blue}{S}\textcolor{red}{S}E\\
= & \textcolor{red}{S}TRA\textcolor{green}{S}\textcolor{blue}{S}E
\end{align*}
das Vertauschen der S ergibt kein neues Wort. Mit 3 Buchstaben sind dies $3!$ Möglichkeiten. Also gibt es
\begin{equation}
\frac{7!}{3!} = \frac{7 \cdot 6 \cdot 5 \cdot 4 \cdot 3 \cdot 2 \cdot 1}{3 \cdot 2 \cdot 1} = 7 \cdot 6 \cdot 5 \cdot 4 = 120 \cdot 7 = 840
\end{equation}
Wir untersuchen nun das folgende Wort (4-mal $S_1$, 4-mal $S_2$ und 2-mal $P$):
\begin{quote}
MISSISSIPPI
\end{quote}
Es gibt
\begin{equation}
\frac{11!}{4! \cdot 4! \cdot 2!} = 34650 \mbox{ Worte}
\end{equation}
\begin{myexample}
Wie gross ist die Wahrscheinlichkeit, beim Werfen von 5 Münzen
\begin{enumerate}[(a)]
\item
dreimal \lq\lq{}Zahl\rq\rq{}
\item
höchstens zweimal \lq\lq{}Zahl\rq\rq{}
\item
mindestens viermal \lq\lq{}Zahl\rq\rq{}
\end{enumerate}
zu erhalten?
\begin{enumerate}[(a)]
\item
\begin{align*}
A = & \{ZZZKK, ZZKKZ, ... \}\\
\longrightarrow |A| = & \frac{5!}{3! \cdot 2!} = \frac{5 \cdot 4}{2!} = 10\\
\longrightarrow P(Z=3) = & \frac{10}{2^5} = \frac{5}{16}
\end{align*}
\item
\begin{align*}
P(Z \leq 2) = & P(Z=2) + P(Z=1) + P(Z=0)\\
= & \frac{10}{2^5} + \frac{5}{2^5} + \frac{1}{2^5} = \frac{16}{2^5} = \frac{1}{2}
\end{align*}
\item
\begin{align*}
P(Z \geq 4) = & P(Z=4) + P(Z=5)\\
= & \frac{5}{2^5} + \frac{1}{2^5} = \frac{6}{2^5} = \frac{3}{16}
\end{align*}
\end{enumerate}
\end{myexample}
Wieviele Möglichkeiten gibt es, um aus einer Menge von $n$ Elementen, deren $k$ $(k < n)$ auszuwählen, wenn die \underline{Reihenfolge nicht wesentlich} ist?\\
Nach dem 2. Zählprinzip gibt es
\begin{equation}
n \cdot (n-1) \cdot (n-2) \cdot (n-3) \cdot ... \cdot (n-k+1)
\end{equation}
Möglichkeiten. Die Reihenfolge ist unwesentlich, d.h.
\begin{quote}
abc = acb = bac = bca = cab = cba
\end{quote}
Dies sind also $3!$ Möglichkeiten. Enthält die Teilmenge $k$ Elemente, so fallen $k!$ Möglichkeiten weg, also gibt es
\begin{equation}
\frac{n \cdot (n-1) \cdot (n-2) \cdot (n-3) \cdot ... \cdot (n-k+1)}{k!}
\end{equation}
Möglichkeiten.
\begin{myexample}
Eine Gruppe von 4 Personen aus einer Menge von 10 Personen auszuwählen. Es gibt
\begin{equation}
\frac{10 \cdot 9 \cdot 8 \cdot 7}{4!} = 210 \mbox{ Möglichkeiten}
\end{equation}
\end{myexample}
Wir erweitern
\begin{equation}
\frac{n \cdot (n-1) \cdot ... \cdot (n-k+1)}{k!} \cdot \textcolor{red}{\frac{(n-k) \cdot (n-k-1) \cdot ... \cdot 3 \cdot 2 \cdot 1}{(n-k) \cdot (n-k-1) \cdot ... \cdot 3 \cdot 2 \cdot 1}} = \frac{n!}{k! \cdot (n-k)!}
\end{equation}
\begin{mydef}
Für $k, n \in \mathbb{N}_{0}$ heisst
\begin{equation}
\binom{n}{k} = \frac{n!}{k!(n-k)!} \quad \mbox{mit } 0 \leq k \leq n
\end{equation}
ein \underline{Binominalkoeffizient} (\lq\lq{}n tief k\rq\rq{}).
\end{mydef}
\begin{myexample}
\begin{equation}
\binom{5}{2} = \frac{5!}{2! \cdot 3!} = \frac{5 \cdot 4}{2 \cdot 1} = 10
\end{equation}
\begin{equation}
\binom{11}{11} = \frac{11!}{11! \cdot 0!} = 1
\end{equation}
\begin{equation}
\binom{7}{0} = \frac{7!}{0! \cdot 7!} = 1
\end{equation}
\end{myexample}
Der Name stammt vom Binom. Der \underline{binomische Lehrsatz}
\begin{equation}
(a + b)^n = \sum^n_{k=0}{\binom{n}{k} a^{n-k} b^k}, n \in \mathbb{N}
\end{equation}
sagt aus, wie Binome berechnet werden.
\begin{myexample}
\begin{equation}
(a+b)^4 = \binom{4}{0} a^4 + \binom{4}{1} a^3b + \binom{4}{2} a^2b^2 + \binom{4}{3} ab^3 + \binom{4}{4} b^4
\end{equation}
\end{myexample}
Also stehen die Binominalkoeffizienten im Pascal\rq{}schen Dreieck.
\begin{mydef}
\underline{4. Zählprinzip:} Eine Menge mit $n$ Elementen besitzt $\binom{n}{k}$ Teilmengen mit genau $k$ $(k \leq n)$ Elementen.\\\\
Es gibt $\binom{n}{k}$ \underline{Kombinationen}, um aus einer Menge mit $n$ Elementen eine Teilmenge mit $k$ Elementen auszuwählen, wenn die Reihenfolge unwesentlich ist.
\end{mydef}
\subsection{Eigenschaften der Binominalkoeffizienten}
\begin{enumerate}
\item
symmetrisch
\begin{equation}
\binom{n}{k} = \binom{n}{n-k}
\end{equation}
\item
\begin{equation}
\binom{n}{0} = \binom{n}{n} = 1
\end{equation}
\item
\begin{equation}
\binom{n}{1} = \binom{n}{n-1} = n
\end{equation}
\item
\begin{equation}
\binom{n}{k} + \binom{n}{k+1} = \binom{n+1}{k+1}
\end{equation}
\end{enumerate}
Für die praktische Rechnung ist
\begin{equation}
\binom{n}{k} = \frac{n!}{k! (n-k)!}
\end{equation}
zu umfangreich. Aber durch Kürzen wird
\begin{equation}
\binom{n}{k} = \frac{n \cdot (n-1) \cdot (n-2) \cdot ... \cdot (n-k+1)}{k!}
\end{equation}
d.h. im Zähler und im Nenner stehen $k$ Faktoren. Also
\begin{equation}
\binom{9}{4} = \frac{}{4!}
\end{equation}
also auch 4 Faktoren im Zähler, d.h.
\begin{equation}
\binom{9}{4} = \frac{9 \cdot 8 \cdot 7 \cdot 6}{4!} = \frac{9 \cdot 8 \cdot 7 \cdot 6}{4 \cdot 3 \cdot 2 \cdot 1} = 9 \cdot 7 \cdot 2 = 126
\end{equation}
\section{Urnenmodell 2}
MIt Hilfe der Binominalkoeffizienten können wir das Ziehen von Kugeln aus einer Urne schneller berechnen. Sind in einer Urne 4 rote und 5 blaue Kugeln und werden ohne Zurücklegen Kugeln gezogen, so ist die Wahrscheinlichkeit zwei rote und drei blaue Kugeln zu ziehen
\begin{equation}
p = \frac{\binom{4}{2} \cdot \binom{5}{3}}{\binom{9}{5}} = \frac{6 \cdot 10}{\frac{9 \cdot 8 \cdot 7 \cdot 6}{4 \cdot 3 \cdot 2 \cdot 1}} = \frac{6 \cdot 10}{126} = \frac{10}{21}
\end{equation}
Allgemein: In einer Urne sind R rote und B blaue Kugeln. Es werden x rote und y blaue ohne zurücklegen gezogen. Dann ist
\begin{equation}
p = \frac{\binom{R}{x} \cdot \binom{B}{y}}{\binom{R+B}{x+y}}
\end{equation}
\begin{myexample}
In einer Kiste sind 10 Werkstücke, wovon 4 fehlerhaft sind. Wie gross ist die Wahrscheinlichkeit, dass sich unter 2 zufällig gewählten Werkstücken 2 fehlerhafte befinden?
\begin{equation}
P(F=2) = \frac{\binom{4}{2} \cdot \binom{6}{0}}{\binom{10}{2}} = \frac{6 \cdot 1}{45} = \frac{2}{15}
\end{equation}
\end{myexample}
Ziehen wir mit Zurücklegen wieder wie vorhin zwei rote und drei blaue aus einer Urne mit 4 roten und 5 blauen Kugeln
\\\\TODO: Grafik\\\\
Die Kugeln rot-rot-blau-blau-blau können auf $\frac{5!}{2! \cdot 3!}$ Arten angeordnet werden. Aber $\frac{5!}{2! \cdot 3!} = \binom{5}{2}$, also
\begin{equation}
p = \binom{5}{2} (\frac{4}{9})^2 (\frac{5}{9})^3
\end{equation}

\chapter{Verteilungen}
\section{Zufallsvariablen}
Werfen wir zwei Münzen und fragen, wie oft \lq\lq{}Zahl\rq\rq{} vorgekommen ist
\begin{center}\begin{tabular}{c | c c c c}
A & KK & KZ & ZK & ZZ\\
\hline
m & 0 & 1 & 1 & 2
\end{tabular}\end{center}
so schreiben wir für die Wahrscheinlichkeit $P(Z=m)$, also zum Beispiel
\begin{equation}
P(Z=1) = \frac{2}{4} = \frac{1}{2}
\end{equation}
Werfen wir zwei Münzen und fragen wir nach den Möglichkeiten die Augensumme A zu würfeln,
\begin{center}\begin{tabular}{c | c c c c c c c c c c c}
A & 2 & 3 & 4 & 5 & 6 & 7 & 8 & 9 & 10 & 11 & 12\\
\hline
m & 1 & 2 & 3 & 4 & 5 & 6 & 5 & 4 & 3 & 2 & 1
\end{tabular}\end{center}
so schreiben wir $P(A=m)$ für die Wahrscheinlichkeit, also zum Beispiel
\begin{equation}
P(A=8) = \frac{5}{36}
\end{equation}
Wir haben also jedem Ereignis A des Stichprobenraums $\Omega$ eine reelle Zahl zugeordnet.
\begin{mydef}
Eine \underline{Zufallsvariable Z} ist eine Funktion
\begin{equation}
Z: \Omega \quad \longrightarrow \quad \mathbb{R}
\end{equation}
Als \underline{Verteilung} oder \underline{Wahrscheinlichkeitsfunktion} einer Zufallsvariablen Z bezeichnen wir
\begin{equation}
P(Z=k)
\end{equation}
\end{mydef}
\begin{myexample}
\begin{enumerate}
\item
Zufallsvariable Z beim Werfen zweier Münzen ist
\begin{equation}
Z: \Omega \quad \longrightarrow \{0,1,2\}
\end{equation}
wenn wir nach der Anzahl \lq\lq{}Zahl\rq\rq{} fragen. Die Verteilung ist dann
\begin{center}\begin{tabular}{c | c c c}
k & 0 & 1 & 2\\
\hline
$P(Z=k)$ & $\frac{1}{4}$ & $\frac{1}{2}$ & $\frac{1}{4}$
\end{tabular}\end{center}
\item
Bei einem Spiel mit 2 Würfeln beträgt der Gewinn 10.- für eine Doppelsechs und 2.- für eine Sechs. Dann ist
\begin{equation}
Z: \Omega \quad \longrightarrow \{10,2,0\}
\end{equation}
mit der Verteilung
\begin{center}\begin{tabular}{c | c c c}
k & 10 & 2 & 0\\
\hline
$P(Z=k)$ & $\frac{1}{36}$ & $\frac{10}{36}$ & $\frac{25}{36}$
\end{tabular}\end{center}
denn eine Sechs erhalten wir mit
\begin{quote}
$(6,x)$, wobei $x \in \{1,2,...,5\}$ oder
$(x,6)$, wobei $x \in \{1,2,...,5\}$
\end{quote}
\end{enumerate}
\end{myexample}
Verteilungen einer Zufallsvariable werden graphisch dargestellt
\\\\TODO: Grafik\\\\
\begin{mydef}
Nimmt eine Zufallsvariable Z die Werte $k_1, k_2, k_3, ..., k_n$ mit den Wahrscheinlichkeiten $p_1, p_2, p_3, ..., p_n$ an, so heisst
\begin{align*}
E(Z) = & k_1p_1 + k_2p_2 + k_3p_3 + ... + k_np_n\\
= & k_1 P(Z=k_1) + k_2 P(Z=k_2) + ... + k_n P(Z=k_n)\\
= & \sum_{i=1}^n{k_i P(Z=k_i)}
\end{align*}
der \underline{Erwartungswert}.
\end{mydef}
\begin{myexample}
Erwartungswert beim Werfen eines Würfels
\begin{center}\begin{tabular}{c | c c c c c c}
k & 1 & 2 & 3 & 4 & 5 & 6\\
\hline
$P(Z=k)$ & $\frac{1}{6}$ & $\frac{1}{6}$ & $\frac{1}{6}$ & $\frac{1}{6}$ & $\frac{1}{6}$ & $\frac{1}{6}$
\end{tabular}\end{center}
ist dann
\begin{equation}
E(Z) = 1 \cdot \frac{1}{6} + 2 \cdot \frac{1}{6} + ... + 6 \cdot \frac{1}{6} = \frac{1}{6}(1+2+...+6) = \frac{21}{6} = 3.5
\end{equation}
\end{myexample}
Wir finden sofort, dass der \underline{Erwartungswert linear} ist.
\begin{equation}
E(X+Y) = E(X) + E(Y)
\end{equation}
\begin{myexample}
Wie gross ist der Erwartungswert beim Werfen von 4 Würfeln?\\
Es ist $E(Z) = 3.5$ für einen Würfel und dadurch wird
\begin{equation}
E(x) = 4 \cdot 3.5 = 14
\end{equation}
der Erwartungswert (für die Augensumme) bei 4 Würfeln.
\end{myexample}

\section{Bernoulli-Experiment}
(Jacob I. Bernoulli von Basel, 1654 bis 1705, Verfasser von \lq\lq{}ars conjectandi\rq\rq{}, die Kunst des Zufalls)\\\\
Wernn wir einen Würfel 100-mal werfen und fragen wann zum ersten mal die Sechs aufgetreten ist, wenn wir eine Münze fünfzigmal werden und fragen wie oft \lq\lq{}Zahl\rq\rq{} vorgekommen ist, wenn wir aus einer Sendung 100 Werksücke wählen und fragen, wieviele davon fehlerhaft sind, so machen wir ein Zufallsexperiment mit $n$ Versuchen.
\begin{mydef}
In einem Experiment werden $n$ Versuche durchgeführt, wobei bei jedem Versuch das Ereignis $A$ mit der Wahrscheinlichkeit $p$ eintreten oder mit der Wahrscheinlichkeit $q = 1-p$ nicht eintreten kann.\\
Sind diese Verscuhe unabhängig, so heisst diese Versuchsreihe vom Umfang $a$ ein \underline{Bernoulli-Experiment}.
\end{mydef}
Wir können also das Experiment direkt mit den Wahrscheinlichkeiten
\begin{quote}
W\rq{}keit: qqpqppq...q\\
binär: 00101001...1
\end{quote}
beschreiben und sprechen von einer \underline{Bernoulli-Kette}.
\begin{mydef}
Eine Zufallsvariable $Z$, welche die Anzahl der notwendigen Schritte bis zum erstmaligen Eintreten des Ereignis $A$ mit Wahrscheinlichkeit $p = P(A)$ in einem Bernoulli-Experiment bestimmt, heisst \underline{geometrisch verteilt}.
\end{mydef}
Es ist also
\begin{equation}
P(Z=k) = (1-p)^{k-1} \cdot p
\end{equation}
Weiter ist also
\begin{equation}
E(Z) = \sum_{k=1}^{\infty}{k \cdot (1-p)^{k-1} \cdot p}
\end{equation}
mit $q = 1-p$ wird
\begin{equation}
E(Z) = p \cdot \sum_{k=1}^{\infty}{k \cdot q^{k-1}} = p \cdot (1 + 2q + 3q^2 + ...)
\end{equation}
Es ist $1, q, q^2, q^3, ...$ eine geometrische Folge mit der Teilsumme
\begin{equation}
s_n = 1 + q + q^2 + q^3 + ... + q^{n-1} = \frac{q^{n-1} - 1}{q - 1}
\end{equation}
Damit wird die \underline{geometrische Reihe}
\begin{equation}
1 + q + q^2 + q^3 + ... = \lim_{n \to \infty}{\frac{q^{n-1}-1}{q-1}} = \frac{-1}{q-1}
\end{equation}
falls $-1 < q < 1$, was für die Wahrscheinlichkeit $q$ zutrifft. Also ist
\begin{equation}
1 + q + q^2 + q^3 + q^4 + ... = \frac{1}{1-q}
\end{equation}
So wird
\begin{align*}
E(Z) = & p (1 + q + q^2 + q^3 + ...) + p ( q + 2q^2 + ...)\\
= & p(1 + q + q^2 + ...) + p(q + q^2 + q^3 + ...) + p (q^2 + 2q^3 + 3q^4 + ...)\\
= & p(1 + q + q^2 + ...) + p(q + q^2 + q^3 + ...) + p (q^2 + q^3 + ...) + p ( q^3 + 2 q^4 + ...)\\
= & p \cdot \frac{1}{1-q} + pq\frac{1}{1-q} + pq^2\frac{1}{1-q} + pq^3\frac{1}{1-q} + ...\\
= & \frac{1}{1-q} ( p + pq + pq^2 + pq^3 + ...)\\
= & \frac{1}{1-q} \cdot \frac{p}{1-q}\\
= & \frac{p}{(1-q)^2} \mbox{ und 1-q=p}
\end{align*}
also
\begin{equation}
E(Z) = \frac{p}{p^2} = \frac{1}{p}
\end{equation}
Beim Würfeln ist also
\begin{equation}
E(Z) = \frac{1}{\frac{1}{6}} = 6
\end{equation}
Das bedeutet also, dass man im Schnitt 6-mal würfeln muss, damit man eine 6 würfelt.
\begin{mydef}
Eine Zufallsvariable Z für welche
\begin{equation}
P(Z=x) = \frac{\binom{R}{x} \cdot \binom{B}{y}}{\binom{R+B}{x+y}}
\end{equation}
gilt, heisst \underline{hypergeometrisch verteilt}. Es ist
\begin{equation}
E(Z) = (x+y) \cdot \frac{B}{B+R}
\end{equation}
wenn $Z$ die Anzahl blauer Kugeln bestimmt.
\end{mydef}
\section{Varianz}
Wir betrachten zwei Zufallsvariablen $X$ und $Y$ mit
\begin{enumerate}
\item
X:
\begin{center}\begin{tabular}{c | c c c}
k & -1 & 0 & 1\\
\hline
$P(Z=k)$ & $\frac{1}{4}$ & $\frac{1}{2}$ & $\frac{1}{4}$
\end{tabular}\end{center}
\item
Y:
\begin{center}\begin{tabular}{c | c c c c c}
m & -4 & -2 & 0 & 2 & 4\\
\hline
$P(Z=m)$ & $\frac{1}{4}$ & $\frac{1}{5}$ & $\frac{1}{10}$ & $\frac{1}{5}$ & $\frac{1}{4}$
\end{tabular}\end{center}
\end{enumerate}
besitzen die Verteilung
\\\\TODO: Grafik\\\\
und die Erwartungswerte
\begin{equation}
E(x) = (-1) \cdot \frac{1}{4} + 0 \cdot \frac{1}{2} + 1 \cdot \frac{1}{4} = 0
\end{equation}
\begin{equation}
E(x) = (-4) \cdot \frac{1}{4} + (-2) \cdot \frac{1}{5} + 0 \cdot \frac{1}{10} + 2 \cdot \frac{1}{4} + 4 \cdot \frac{1}{4} = 0
\end{equation}
Kennen wir also nur die Erwartungswerte, so wissen wir nicht wie die Werte verteilt sind.\\
Wir wählen deshalb für jeden Wert $x_i$ die Abweichung von $E(x)$ und bestimmen dadurch den Durchschnitt. Bis $X$ wäre dies also
\begin{equation}
\frac{(0+1) + (0-0) + (0-1)}{3} = 0 \mbox{ !?}
\end{equation}
Besser ist es also mit
\begin{equation}
\frac{|0+1| + |0-0| + |0-1|}{3} = \frac{2}{3}
\end{equation}
Historisch gesehen war es einfacher mit $\sqrt{a^2}$ zu rechnen, anstatt mit $|a|$. Deshalb definieren wir unser \underline{Mass für die Streuung} mit
\begin{equation}
\sigma^2 := Var(Z) := \sum_{i=1}^n{(k_i - E(Z))^2 \cdot p_i}
\end{equation}
die \underline{Varianz} der Zufallsvariablen $Z$, wenn
\begin{equation}
P_i = P(Z=k_i)
\end{equation}
$\sigma$ heisst \underline{Standardabweichung}.\\
Im Beispiel wird so
\begin{equation}
Var(x) = \sigma_x^2 = \frac{1}{4}(0-(-1))^2 + \frac{1}{2}(0-0)^2 + \frac{1}{4}(0-1)^2 = \frac{1}{2}
\end{equation}
also
\begin{equation}
\sigma = \sqrt{\frac{1}{2}} = 0.707
\end{equation}
und weiter
\begin{align*}
Var(Y) = & \sigma^2_y = (\frac{1}{4}(0-(-4))^2 + \frac{1}{5}(0-(-2))^2 + \frac{1}{10}(0-0)^2) \cdot 2\\
= & (4 + \frac{4}{5}) \cdot 2 = \frac{48}{5} = 9.6
\end{align*}
also
\begin{equation}
\sigma = \sqrt{9.6} = 3.1
\end{equation}
Bei einer geometrisch verteilten Zufallsvariable wird
\begin{equation}
Var(Z) = \frac{q}{p^2}
\end{equation}
also bei \lq\lq{}Eine mit Weile\rq\rq{} ist
\begin{equation}
Var(Z) = \frac{\frac{5}{6}}{(\frac{1}{6})^2} = \frac{5}{6} \cdot \frac{36}{1} = 30
\end{equation}
und damit wird $\sigma = 5.47$


\begin{myexample}
\end{myexample}
\begin{mydef}
\end{mydef}
\end{document}