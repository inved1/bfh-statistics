\documentclass{article}

\usepackage[utf8]{inputenc}
\title{Laplace-Würfel}
\author{Simon Krenger}

\begin{document}
\section{Augensummen}
\subsection{2 Würfel}
\begin{tabular}{c | c}
Summe & Wahrscheinlichkeit \\
\hline
2 & 1/36 = 2,78 \% \\
3 & 2/36 = 5,56 \%\\
4 & 3/36 = 8,33 \%\\
5 & 4/36 = 11,11 \%\\
6 & 5/36 = 13,89 \%\\
7 & 6/36 = 16,67 \%\\
8 & 5/36 = 13,89 \%\\
9 & 4/36 = 11,11 \%\\
10 & 3/36 = 8,33 \%\\
11 & 2/36 = 5,56 \%\\
12 & 1/36 = 2,78 \%
\end{tabular}

Denn es ist:\\
2 = 1/1\\
3 = 1/2 = 2/1\\
4 = 1/3 = 2/2 = 3/1\\
5 = 1/4 = 2/3 = 3/2 = 4/1\\
6 = 1/5 = 2/4 = 3/3 = 4/2 = 5/1\\
7 = 1/6 = 2/5 = 3/4 = 4/3 = 5/2 = 6/1\\
8 = 2/6 = 3/5 = 4/4 = 5/3 = 6/2\\
9 = 3/6 = 4/5 = 5/4 = 6/3\\
10 = 4/6 = 5/5 = 6/4\\
11 = 5/6 = 6/5\\
12 = 6/6
\pagebreak
\subsection{3 Würfel}

\begin{tabular}{c | c}
Summe & Wahrscheinlichkeit \\
\hline
3 & 1/216\\
4 & 2/216\\
5 & 6/216\\
6 & 10/216\\
7 & 15/216\\
8 & 21/216\\
9 & 25/216\\
10 & 27/216\\
11 & 27/216\\
12 & 25/216\\
13 & 21/216\\
14 & 15/216\\
15 & 10/216\\
16 & 6/216\\
17 & 3/216\\
18 & 1/216\\
\end{tabular}

Denn es ist (Achtung, nicht alle Permutationen):\\
3 = 1/1/1\\
4 = 1/1/2\\
5 = 1/1/3 = 2/2/1\\
6 = 1/1/4 = 1/2/3 = 2/2/2\\
7 = 1/1/5 = 2/2/3 = 3/3/1 = 1/2/4\\
8 = 1/1/6 = 2/3/3 = 4/3/1 = 1/2/5 = 2/2/4\\
9 = 6/2/1 = 4/3/2 = 3/3/3 = 2/2/5 = 1/3/5 = 1/4/4\\
10 = 6/3/1 = 6/2/2 = 5/3/2 = 4/4/2 = 4/3/3 = 1/4/5\\
11 = 6/4/1 = 1/5/5 = 5/4/2 = 3/3/5 = 4/3/4 = 6/3/2\\
12 = 6/5/1 = 4/3/5 = 4/4/4 = 5/2/5 = 6/4/2 = 6/3/3\\
13 = 6/6/1 = 5/4/4 = 3/4/6 = 6/5/2 = 5/5/3\\
14 = 6/6/2 = 5/5/4 = 4/4/6 = 6/5/3\\
15 = 6/6/3 = 6/5/4 = 5/5/5\\
16 = 6/6/4 = 5/5/6\\
17 = 6/6/5\\
18 = 6/6/6
\end{document}
