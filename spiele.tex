\documentclass{article}
\usepackage{amsmath}
\usepackage{amsthm}
\usepackage{amssymb}
\usepackage[utf8]{inputenc} 
\usepackage{graphicx}
\usepackage{mathtools}
\usepackage{floatrow}
\usepackage{enumerate}

\title{Spiele}
\author{Simon Krenger}

\begin{document}
\section{Lotto}
Schweizer Lotto (6 aus 42 und 6 Glückszahlen)\\
\underline{Ein Sechser}: Günstiger Fall 1
\begin{equation}
p = \frac{1}{\binom{42}{6}} = 1,9 * 10^{-7}
\end{equation}
\underline{Ein Fünfer}: 5 der 6 und noch eine zusätzliche, aber nicht diejenige die einen Sechser gibt
\begin{equation}
p = \frac{\binom{6}{5} \cdot 36}{\binom{42}{6}} = 4,1 * 10^{-5}
\end{equation}
\underline{Ein Vierer}:
\begin{equation}
p = \frac{\binom{6}{4} \cdot \binom{36}{2}}{\binom{42}{6}} = 0.0018
\end{equation}

\section{Jass}
\underline{Vier Bauern (Asse, Könige, Damen, ...)} : 4 Bauern und noch 5 Karten vom Rest
\begin{equation}
p = \frac{\binom{32}{5}}{\binom{36}{9}} = 0.002
\end{equation}
\underline{Ein Dreiblatt vom Herz-As}: Das Dreiblatt und dann noch 6 vom Rest, aber nicht Herz-Bube
\begin{equation}
p = \frac{\binom{33-1}{6}}{\binom{36}{9}} = 0.009
\end{equation}
\underline{Ein Dreiblatt vom Pik-König}: Nicht das As und 10 von Pik	
\begin{equation}
p = \frac{\binom{31}{6}}{\binom{36}{9}} = 0.007
\end{equation}
\underline{50 vom Kreuz-König weisen}: Nicht das As und nicht die 9
\begin{equation}
p = \frac{\binom{30}{5}}{\binom{36}{9}}
\end{equation}
\underline{100 vom Kreuz-König weisen}: Nicht das As und nicht die 8
\begin{equation}
p = \frac{\binom{29}{4}}{\binom{36}{9}} = \frac{117}{463670} = 0.000252
\end{equation}

\section{Poker}
Jeder Spieler erhält 5 von 52 Karten. Das sind $\binom{52}{5} = 2 598 960$ mögliche Fälle.\\\\
\underline{Royal Flush}: Karten A, K, Q, J, 10 der gleichen Farbe
\begin{equation}
p = \frac{4}{\binom{52}{5}} = 0.0000015
\end{equation}
\underline{Straight Flush}: Fünf aufeinanderfolgende Karten derselben Farbe. Es gibt 9 solche Strassen pro Farbe
\begin{equation}
p = \frac{4 \cdot 9}{\binom{52}{5}} = 0.000014
\end{equation}
\underline{Four of a kind (Vierling, Poker)}: Vier gleiche Karten
\begin{equation}
p = \frac{13 \cdot \binom{48}{1}}{\binom{52}{5}} = 0.00024
\end{equation}
\underline{Full House}: Drilling und ein Paar (13 Karten zur Auswahl, also $\binom{4}{2} = 6$ Möglichkeiten. Für das Tripel haben wir dann noch 12 Kartentypen zur Auswahl und so $\binom{4}{3} = 4$ Möglichkeiten. Es gibt somit $13 \cdot 6 \cdot 12 \cdot 4 = 3744$ verschiedene Full Houses.
\begin{equation}
p = \frac{13 \cdot 6 \cdot 12 \cdot 4}{\binom{52}{5}} = \frac{3744}{\binom{52}{5}} = 0.0014
\end{equation}
\underline{Flush}: Fünf Karten derselben Farbe
\begin{equation}
p = \frac{4 \cdot (\binom{13}{5} - 1 - 9)}{\binom{52}{5}} = 0.002
\end{equation}
\underline{Straight}: Fünf aufeinanderfolgende Karten mit verschiedenen Farben. Es gibt 10 solche Straights pro Figur und pro Platz eine der 4 Farben, also $10 \cdot 4^5$ wovon die Flushes zu subtrahieren sind.
\begin{equation}
p = \frac{10 \cot 4^5 - 40}{\binom{52}{5}} = \frac{10 200}{\binom{52}{5}} = 0.0039
\end{equation}
\underline{Drilling}: Nicht vierte der Drillingskarten und kein Paar mehr (Full House)
\begin{equation}
p = \frac{13 \cdot \binom{4}{3} \cdot (\binom{48}{2} - 12 \binom{4}{2})}{\binom{52}{5}} = 0.021
\end{equation}
\underline{Zwei Paare}
\begin{equation}
p = \frac{\binom{13}{2} \binom{4}{2} \binom{4}{2} \cdot 44}{\binom{52}{5}} = 0.047
\end{equation}
\underline{Ein Paar}: 13 Kartentypen für das Paar mit 2 von 4 Karten, also $13 \cdot \binom{4}{2}$ Möglichkeiten. Die restlichen so, dass kein Full House und kein zweites Paar entsteht.
\begin{equation}
p = \frac{13 \cdot \binom{4}{2} \binom{12}{3} \cdot 4^3}{\binom{52}{5}} = 0.42
\end{equation}
\underline{High Card}
\begin{equation}
p = \frac{1 302 540}{\binom{52}{5}} = 0.501
\end{equation}
\end{document}
