\documentclass{article}
\usepackage{amsmath}
\usepackage{amsthm}
\usepackage{amssymb}
\usepackage[utf8]{inputenc} 
\usepackage{graphicx}
\usepackage{mathtools}
\usepackage{floatrow}
\usepackage{enumerate}

\newtheorem{mydef}{Definition}

\title{Formelsammlung}
\author{Simon Krenger}

\begin{document}
\section{Defintionen}
\begin{mydef}
Die Menge
\begin{equation}
\Omega := \{ \omega_1, \omega_2, \omega_3, ..., \omega_n \}
\end{equation}
heisst \underline{Stichprobenraum} (Ereignisraum), wenn jedem Versuchsausgang höchstens ein Element $\omega_i$ aus $\Omega$ zugeordnet ist.
\end{mydef}
\begin{mydef}
Jede Teilmenge von $\Omega$ heisst \underline{Ereignis}. Die leere Menge $\emptyset$ heisst \underline{unmögliches Ereignis} und $\Omega$ heisst \underline{sicheres Ereignis}.\\
Enthält ein Ereignis $E = \{ a \}$ nur ein einziges Element, so heisst $E$ ein \underline{Elementarereignis}.
\end{mydef}
\begin{mydef}
\underline{1. Zählprinzip}: Es gibt $n^k$ Möglichkeiten um $n$ Elemente auf $k$ Plätze (in Gruppen zu $k$ Elementen) zu verteilen.
\end{mydef}
\begin{mydef}
Wir nennen $P(\Omega)$ den \underline{Ereignisraum}.
\end{mydef}
\begin{mydef}
Zwei Ereignisse heissen (stochastisch) \underline{unabhängig}, wenn
\begin{equation}
P(A \cap B) = P(A) \cdot P(B)
\end{equation}
\end{mydef}
\begin{mydef}
\underline{2. Zählprinzip:} Wählen wir $k$ aus $n$ Elementen, so gibt es also
\begin{equation}
n (n-1) (n-2) ... (n-k+1)
\end{equation}
Möglichkeiten, wenn die Reihenfolge wesentlich ist.
\end{mydef}
\begin{mydef}
\underline{Gegenwahrscheinlichkeit}: Es ist also
\begin{equation}
P(\overline{A}) = 1 - P(A)
\end{equation}
heisst die Gegenwahrscheinlichkeit von A.
\end{mydef}
\begin{mydef}
Ist $n \in \mathbb{N}$, so heisst
\begin{equation}
n! := n \cdot (n-1) \cdot (n-2) \cdot ... \cdot 3 \cdot 2 \cdot 1
\end{equation}
die \underline{Fakultät} von $n$. Ausserdem ist $0! = 1$.
\end{mydef}
\begin{mydef}
\underline{3. Zählprinzip:} Es gibt $n!$ Möglichkeiten um $n$ Elemente auf $n$ Plätze zu verteilen.
\end{mydef}
\begin{mydef}
Für $k, n \in \mathbb{N}_{0}$ heisst
\begin{equation}
\binom{n}{k} = \frac{n!}{k!(n-k)!} \quad \mbox{mit } 0 \leq k \leq n
\end{equation}
ein \underline{Binominalkoeffizient} (\lq\lq{}n tief k\rq\rq{}).
\end{mydef}
\begin{mydef}
\underline{4. Zählprinzip:} Eine Menge mit $n$ Elementen besitzt $\binom{n}{k}$ Teilmengen mit genau $k$ $(k \leq n)$ Elementen.\\\\
Es gibt $\binom{n}{k}$ \underline{Kombinationen}, um aus einer Menge mit $n$ Elementen eine Teilmenge mit $k$ Elementen auszuwählen, wenn die Reihenfolge unwesentlich ist.
\end{mydef}
Eigenschaften der Binominalkoeffizienten
\begin{enumerate}
\item
symmetrisch
\begin{equation}
\binom{n}{k} = \binom{n}{n-k}
\end{equation}
\item
\begin{equation}
\binom{n}{0} = \binom{n}{n} = 1
\end{equation}
\item
\begin{equation}
\binom{n}{1} = \binom{n}{n-1} = n
\end{equation}
\item
\begin{equation}
\binom{n}{k} + \binom{n}{k+1} = \binom{n+1}{k+1}
\end{equation}
\end{enumerate}
\begin{mydef}
Eine \underline{Zufallsvariable Z} ist eine Funktion
\begin{equation}
Z: \Omega \quad \longrightarrow \quad \mathbb{R}
\end{equation}
Als \underline{Verteilung} oder \underline{Wahrscheinlichkeitsfunktion} einer Zufallsvariablen Z bezeichnen wir
\begin{equation}
P(Z=k)
\end{equation}
\end{mydef}
\begin{mydef}
Nimmt eine Zufallsvariable Z die Werte $k_1, k_2, k_3, ..., k_n$ mit den Wahrscheinlichkeiten $p_1, p_2, p_3, ..., p_n$ an, so heisst
\begin{align*}
E(Z) = & k_1p_1 + k_2p_2 + k_3p_3 + ... + k_np_n\\
= & k_1 P(Z=k_1) + k_2 P(Z=k_2) + ... + k_n P(Z=k_n)\\
= & \sum_{i=1}^n{k_i P(Z=k_i)}
\end{align*}
der \underline{Erwartungswert}.
\end{mydef}
\begin{mydef}
In einem Experiment werden $n$ Versuche durchgeführt, wobei bei jedem Versuch das Ereignis $A$ mit der Wahrscheinlichkeit $p$ eintreten oder mit der Wahrscheinlichkeit $q = 1-p$ nicht eintreten kann.\\
Sind diese Verscuhe unabhängig, so heisst diese Versuchsreihe vom Umfang $a$ ein \underline{Bernoulli-Experiment}.
\end{mydef}
\begin{mydef}
Eine Zufallsvariable $Z$, welche die Anzahl der notwendigen Schritte bis zum erstmaligen Eintreten des Ereignis $A$ mit Wahrscheinlichkeit $p = P(A)$ in einem Bernoulli-Experiment bestimmt, heisst \underline{geometrisch verteilt}.
\end{mydef}
\begin{mydef}
Eine Zufallsvariable Z für welche
\begin{equation}
P(Z=x) = \frac{\binom{R}{x} \cdot \binom{B}{y}}{\binom{R+B}{x+y}}
\end{equation}
gilt, heisst \underline{hypergeometrisch verteilt}. Es ist
\begin{equation}
E(Z) = (x+y) \cdot \frac{B}{B+R}
\end{equation}
wenn $Z$ die Anzahl blauer Kugeln bestimmt.
\end{mydef}
\begin{mydef}Tritt in einem Bernoulli-Experiment das Ereignis $A$ mit Wahrscheinlichkeit $p$ ein und fragen wir nach der Wahrscheinlichkeit, dass $A$ k-mal in $n$ Versuchen aufgetreten ist, so ist die Wahrscheinlichkeit
\begin{equation}
P(Z=k) = \binom{n}{k} p^kq^{n-k} \quad , q=1-p
\end{equation}
eine \underline{binominalverteilte} Zufallsvariable.
\end{mydef}
\begin{mydef}
Ist $Z$ eine Zufallsvariable, welche die Werte $0, 1, 2, ..., k$ mit der Wahrscheinlichkeit
\begin{equation}
P(Z=k) = e^{-\lambda} \cdot \frac{\lambda^k}{k!}
\end{equation}
annimmt, so heisst $Z$ \underline{poisson-verteilt}.
\end{mydef}
\begin{mydef}
Ist $Z$ eine Zufallsvariable, so heisst
\begin{equation}
F: \mathbb{R} \longrightarrow [0;1] \mbox{ mit } F(x) = P(Z \leq x)
\end{equation}
die Verteilungsfunktion.
\end{mydef}
\begin{mydef}
Ist
\begin{equation}
F(x) = P(Z \leq x) = \int_{-\infty}^{x}{f(t)} \mathrm{d}t
\end{equation}
, so heisst die Funktion $f$ \underline{Dichtefunktion} und $f(x)$ \underline{Dichte}.
\end{mydef}



\section{Laplace-Würfel: Augensummen}
\subsection{2 Würfel}
\begin{tabular}{c | c}
Summe & Wahrscheinlichkeit \\
\hline
2 & 1/36 = 2,78 \% \\
3 & 2/36 = 5,56 \%\\
4 & 3/36 = 8,33 \%\\
5 & 4/36 = 11,11 \%\\
6 & 5/36 = 13,89 \%\\
7 & 6/36 = 16,67 \%\\
8 & 5/36 = 13,89 \%\\
9 & 4/36 = 11,11 \%\\
10 & 3/36 = 8,33 \%\\
11 & 2/36 = 5,56 \%\\
12 & 1/36 = 2,78 \%
\end{tabular}
 \\\\
Denn es ist (mit allen Permutationen):\\
2 = 1/1\\
3 = 1/2 = 2/1\\
4 = 1/3 = 2/2 = 3/1\\
5 = 1/4 = 2/3 = 3/2 = 4/1\\
6 = 1/5 = 2/4 = 3/3 = 4/2 = 5/1\\
7 = 1/6 = 2/5 = 3/4 = 4/3 = 5/2 = 6/1\\
8 = 2/6 = 3/5 = 4/4 = 5/3 = 6/2\\
9 = 3/6 = 4/5 = 5/4 = 6/3\\
10 = 4/6 = 5/5 = 6/4\\
11 = 5/6 = 6/5\\
12 = 6/6

\subsection{3 Würfel}

\begin{tabular}{c | c}
Summe & Wahrscheinlichkeit \\
\hline
3 & 1/216\\
4 & 2/216\\
5 & 6/216\\
6 & 10/216\\
7 & 15/216\\
8 & 21/216\\
9 & 25/216\\
10 & 27/216\\
11 & 27/216\\
12 & 25/216\\
13 & 21/216\\
14 & 15/216\\
15 & 10/216\\
16 & 6/216\\
17 & 3/216\\
18 & 1/216\\
\end{tabular}
 \\\\
Denn es ist (Achtung, nicht alle Permutationen):\\
3 = 1/1/1\\
4 = 1/1/2\\
5 = 1/1/3 = 2/2/1\\
6 = 1/1/4 = 1/2/3 = 2/2/2\\
7 = 1/1/5 = 2/2/3 = 3/3/1 = 1/2/4\\
8 = 1/1/6 = 2/3/3 = 4/3/1 = 1/2/5 = 2/2/4\\
9 = 6/2/1 = 4/3/2 = 3/3/3 = 2/2/5 = 1/3/5 = 1/4/4\\
10 = 6/3/1 = 6/2/2 = 5/3/2 = 4/4/2 = 4/3/3 = 1/4/5\\
11 = 6/4/1 = 1/5/5 = 5/4/2 = 3/3/5 = 4/3/4 = 6/3/2\\
12 = 6/5/1 = 4/3/5 = 4/4/4 = 5/2/5 = 6/4/2 = 6/3/3\\
13 = 6/6/1 = 5/4/4 = 3/4/6 = 6/5/2 = 5/5/3\\
14 = 6/6/2 = 5/5/4 = 4/4/6 = 6/5/3\\
15 = 6/6/3 = 6/5/4 = 5/5/5\\
16 = 6/6/4 = 5/5/6\\
17 = 6/6/5\\
18 = 6/6/6
\pagebreak

\section{Spiele}
\subsection{Lotto}
Schweizer Lotto (6 aus 42 und 6 Glückszahlen)\\
\underline{Ein Sechser}: Günstiger Fall 1
\begin{equation}
p = \frac{1}{\binom{42}{6}} = 1,9 * 10^{-7}
\end{equation}
\underline{Ein Fünfer}: 5 der 6 und noch eine zusätzliche, aber nicht diejenige die einen Sechser gibt
\begin{equation}
p = \frac{\binom{6}{5} \cdot 36}{\binom{42}{6}} = 4,1 * 10^{-5}
\end{equation}
\underline{Ein Vierer}:
\begin{equation}
p = \frac{\binom{6}{4} \cdot \binom{36}{2}}{\binom{42}{6}} = 0.0018
\end{equation}

\subsection{Jass}
\underline{Vier Bauern (Asse, Könige, Damen, ...)} : 4 Bauern und noch 5 Karten vom Rest
\begin{equation}
p = \frac{\binom{32}{5}}{\binom{36}{9}} = 0.002
\end{equation}
\underline{Ein Dreiblatt vom Herz-As}: Das Dreiblatt und dann noch 6 vom Rest, aber nicht Herz-Bube
\begin{equation}
p = \frac{\binom{33-1}{6}}{\binom{36}{9}} = 0.009
\end{equation}
\underline{Ein Dreiblatt vom Pik-König}: Nicht das As und 10 von Pik	
\begin{equation}
p = \frac{\binom{31}{6}}{\binom{36}{9}} = 0.007
\end{equation}
\underline{50 vom Kreuz-König weisen}: Nicht das As und nicht die 9
\begin{equation}
p = \frac{\binom{30}{5}}{\binom{36}{9}}
\end{equation}
\underline{100 vom Kreuz-König weisen}: Nicht das As und nicht die 8
\begin{equation}
p = \frac{\binom{29}{4}}{\binom{36}{9}} = \frac{117}{463670} = 0.000252
\end{equation}

\subsection{Poker}
Jeder Spieler erhält 5 von 52 Karten. Das sind $\binom{52}{5} = 2 598 960$ mögliche Fälle.\\\\
\underline{Royal Flush}: Karten A, K, Q, J, 10 der gleichen Farbe
\begin{equation}
p = \frac{4}{\binom{52}{5}} = 0.0000015
\end{equation}
\underline{Straight Flush}: Fünf aufeinanderfolgende Karten derselben Farbe. Es gibt 9 solche Strassen pro Farbe
\begin{equation}
p = \frac{4 \cdot 9}{\binom{52}{5}} = 0.000014
\end{equation}
\underline{Four of a kind (Vierling, Poker)}: Vier gleiche Karten
\begin{equation}
p = \frac{13 \cdot \binom{48}{1}}{\binom{52}{5}} = 0.00024
\end{equation}
\underline{Full House}: Drilling und ein Paar (13 Karten zur Auswahl, also $\binom{4}{2} = 6$ Möglichkeiten. Für das Tripel haben wir dann noch 12 Kartentypen zur Auswahl und so $\binom{4}{3} = 4$ Möglichkeiten. Es gibt somit $13 \cdot 6 \cdot 12 \cdot 4 = 3744$ verschiedene Full Houses.
\begin{equation}
p = \frac{13 \cdot 6 \cdot 12 \cdot 4}{\binom{52}{5}} = \frac{3744}{\binom{52}{5}} = 0.0014
\end{equation}
\underline{Flush}: Fünf Karten derselben Farbe
\begin{equation}
p = \frac{4 \cdot (\binom{13}{5} - 1 - 9)}{\binom{52}{5}} = 0.002
\end{equation}
\underline{Straight}: Fünf aufeinanderfolgende Karten mit verschiedenen Farben. Es gibt 10 solche Straights pro Figur und pro Platz eine der 4 Farben, also $10 \cdot 4^5$ wovon die Flushes zu subtrahieren sind.
\begin{equation}
p = \frac{10 \cot 4^5 - 40}{\binom{52}{5}} = \frac{10 200}{\binom{52}{5}} = 0.0039
\end{equation}
\underline{Drilling}: Nicht vierte der Drillingskarten und kein Paar mehr (Full House)
\begin{equation}
p = \frac{13 \cdot \binom{4}{3} \cdot (\binom{48}{2} - 12 \binom{4}{2})}{\binom{52}{5}} = 0.021
\end{equation}
\underline{Zwei Paare}
\begin{equation}
p = \frac{\binom{13}{2} \binom{4}{2} \binom{4}{2} \cdot 44}{\binom{52}{5}} = 0.047
\end{equation}
\underline{Ein Paar}: 13 Kartentypen für das Paar mit 2 von 4 Karten, also $13 \cdot \binom{4}{2}$ Möglichkeiten. Die restlichen so, dass kein Full House und kein zweites Paar entsteht.
\begin{equation}
p = \frac{13 \cdot \binom{4}{2} \binom{12}{3} \cdot 4^3}{\binom{52}{5}} = 0.42
\end{equation}
\underline{High Card}
\begin{equation}
p = \frac{1 302 540}{\binom{52}{5}} = 0.501
\end{equation}

\end{document}